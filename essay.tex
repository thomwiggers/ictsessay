\documentclass{article}

\usepackage[backend=biber]{biblatex}
\usepackage[utf8]{inputenc}
\usepackage{hyperref}
\usepackage{microtype}

\addbibresource{essay.biblatex}

\author{Thom Wiggers}
\title{\Huge OSS: Open Source Society? \\ \Large ICT \& Society,\\ \large Radboud University Nijmegen}

\begin{document}
\maketitle

In recent years there has been a growing emphasis on using open source
technologies in the field of software development and IT in general. (Free)
open source software (FOSS) is code which is freely made available so that
anyone may use it and adapt it to their own ends --- within certain limits,
mainly about having to share your modifications with the community. FOSS
products are produced both by a very active community and by large software
vendors, like Google or IBM. Nowadays, it is impossible to ignore products like
Firefox and Chromium browsers and the GNU/Linux operating system, which are
used by millions of people on a daily basis, either directly being used by end
users or by being the platform that is being used by most of the internet's
back end.

Open source has made a significant impact on the IT community, but also on
business and on society as a whole. In this essay I will first explain how open
source became the movement it is today. Then I will explain how it has changed
business models and how the open source movement contributes to modern society.
I will then explain why I personally contribute to open source projects. Concluding
I will present my view on FOSS and its impact on society.

In the early years of software development, software was created by large
companies which had a monopoly in their field, like telephone companies. The
early developers freely shared their software and developed the software in
a open and communal process.\cite{kogut01}

UNIX is an operating system which was developed in that time, mainly by US
company AT\&T. They had shared this operating system with a number of academic
institutions and businesses. The software community then did not restrict
property rights and they did not define authorship. AT\&T then started to claim
it owned UNIX to which many of the institutions and firms had made
contributions. This triggered the software community to start establishing
ground rules for the communal development efforts, which for instance led to
the foundation of the Free Software Foundation by Richard Stallman, who worked
at MIT's Artificial Intelligence Lab at the time.\cite{lerner04} He later quit
working for MIT to work on the Free Software Foundation
full-time.\cite{fsfstallman}

The Free Software Foundation created the General Public Licence and Stallman
started work on the GNU (``GNU's Not UNIX'') operating system. This work
eventually led to the GNU/Linux operating system, which uses GNU software and
a linux kernel.\cite{fsfstallman} Most servers in the world run GNU/Linux
software distributions as their operating system; linux is a key part of the
internet infrastructure. The General Public Licence (GPL) is the licence with
which the GNU project and the Linux Foundation ship its software and the
licence is used by most open source projects in the world.\cite{lerner04} The
GPL licence is a fairly strict one: in addition to requiring the availability
of the source of modified versions, if one wants to use GPL code in another
piece of code they should make that code GPL-licenced as well.\cite{gpl}

When in the early 1990s the internet became widespread, open source
contributions increased tremendously. Linux, by Linus Torvalds, was one of the
projects that emerged around that time. Different licences also emerged,
oftentimes much more liberal than the GPL, not requiring that open source code
could only be used in open source applications.

Economist were initially baffled that people would volunteer their time to work
on open source projects, which would then be used by people and companies for
free. When they looked closer they noticed that there are different benefits
than monetary ones that motivate the contributing developers and companies.
While this includes rewards that can more or less be expressed in money, like
additional experience or the chance to get access to venture capital, intrinsic
motivation like pleasure and satisfaction is very much a large
factor.\cite{lerner04}

Reasons for companies to participate in open source software development can be
the result of their adapting the code to their needs and fixing bugs, then
releasing these changes back into the wild. They thus can benefit from using
the software libraries that are being maintained by other people, but still be
able to customise them to their needs.\cite{lerner04} But this is not the only
way companies contribute: sometimes companies have released large applications
that were initially proprietary to the public. IBM, for instance, made their
Eclipse Integrated Development Environment, a tool used when developing
software, open source. The value of the source code was estimated at roughly 40
million US Dollar. While intuitively it would seem a stupid move to give away
40 million US Dollar, it has greatly increased the popularity of the product
and increased the market for IBMs complementary products.\cite{fitzgerald06} 

Another way companies might benefit from open source is the increased
involvement of the community and other companies in open source applications.
Social interaction and flat organisations common in FOSS projects are a huge
driver for innovation. \cite{conway09} 

But these arguments which are based on money, don't rule out an ideological
view from management. A lot of participants in open source of course do so for
ideological reasons: contributing to society or to the community of
programmers. This is a community from which programmers and companies often
borrow techniques, software libraries and platform applications and a lot of
people feel that they can help other people by fixing or adding things.

Open source software has also allowed a lot of people who are not part of the
programming world to benefit. For instance the ``One Laptop Per
Child''-project, which aimed to supply millions of kids in the third world with
cheap laptops, uses a GNU/Linux deriviated operating system which they call
Sugar. They have customised this OS to work as good as possible with the
hardware they distribute and include software specifically aimed at kids who
don't have much experience with computers.\cite{sugar} This would have been
much harder and costlier to do if they could not have built on the work of the
Linux OS and many volunteers. 

The lack of license fees also greatly beneficial to countries which are less
developed than European ones. Labour in developing countries often is very
cheap when compared to licencing fees, while in the western world it often is
the reverse. This enables those developing countries to implement a very cheap
IT infrastructure supported by the open source software
community.\cite{ghosh03}

Open source applications are also used to crowdsource crisis information.
Developers from Kenya developed this application after election
turmoils\cite{ushahidi}, but has been used in a lot of cases, from snow plowing
to supplying information about the Haiti earthquakes.\cite{shirkyTED}

I am a part of the open source community as well. I use a lot of open source
products both personally and professionally: for instance, I run the Ubuntu
\textsc{Gnome} GNU/Linux distribution, and professionally, in my work as a web
application developer, I use open source tools, like Eclipse, open source
programming languages, like python, and I use a lot of open source software
libraries, like jQuery. I like, and my bosses like this because it saves them
time, to reuse components that other people have written for me. In return,
I feel it is my moral obligation to supply bug reports and bug fixes to these
projects whenever I encounter problems. Often, I have to fix the bug or add the
feature anyway, so why not contribute it back to the community so they don't
have to reinvent the wheel again? I'm profiting as much as they are, since
I can use their bugfixes.

As a hobbist I also sometimes write code. I, for instance, operate an
IRC\footnote{Internet Relay Chat} bot on several chat networks. It's very
unlikely this will ever generate me money, so I've put the source code online
for three reasons: the first is being able to refer other people to the
codebase when they want me to change something (``do it yourself, and I'll add
it to the running bot''), the second is ego. It just feels good to have a body
of work to show off. The third reason is I like being able to use the search
for snippets of code in other people's applications so I can learn from them:
why not do the same? I've for instance learnt a lot of \LaTeX, the markup
language in which this document is formatted, by looking at other people's
documents. Seeing larger, working examples enabled me to structure my
documents. This is why I'm also putting this document online, its source code
is available at \url{https://github.com/thomwiggers/ictsessay/}. I like to
think that with these small acts, developers like me are keeping the open
source community alive.

I think that free open source software is having a significant impact on both
business and on society. It has certainly brought lots of developers together
in various communities, but I also showed it can be a valuable resource in, for
instance, third world countries. Personally, I like helping people and I try to
contribute where I can, especially if I have to do the work anyway: I might as
well give other people a hand as well in that case. 

In business, open source can be a katalyst for innovation. For instance the
Android mobile phone operating system was developed in the open, but has shaped
how smartphonse look and act today, which is nothing like they used to 5 years
ago. As an IT platform which enables many other functions in business, open
source will be here to stay. 

Should we do more with open source software? From a software developer's view
I would say we should. Open source software has taught me a lot about how
programming and projects work, which would have been hard or even impossible to
gain knowledge about if all software development were behind closed doors.
I also am a great believer in the power of the community: someone somewhere
will have had your problem already.
 
From a practival point of view I think there will still be room for proprietary
software. Not necessary specific functionality however should remain
proprietary, mainly how it is combined into a useful program. Software should
be using a lot of open source components. This benefits the companies, not
having to reinvent the wheel for every minor thing, but this also benefits the
hobbist: they are also able to use industry-tested tools and components to be
able to construct their own combinations. Using this synergy, I think the
software producing community will be able to advance and push the boundary,
extending functionality and making life easier and increasing welfare for
everyone.


\printbibliography 

\end{document}
